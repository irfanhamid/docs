\documentclass{article}
\usepackage{fullpage}

\title{Proposed changes to manuscript}
\author{Irfan Hamid}

\begin{document}
\maketitle

\section{Overall picture}
There are three main changes to the big picture that need to be
effected.

\subsection{Placement of approach in the process}
Clarify that the approach given is a proposal to change the style of
engineering real-time systems.

There is no mention of the ``process'', put the proposed approach
\emph{somewhere} within the SDLC, is it requirements capture? Is it
specification? Is it design? Is it coding? Is it testing?

What other areas of the SDLC other than design can be potentially
impacted by this technique?

Also, explicitly state where in the process does AADL fit in?

\subsection{Bibliographic improvements}
Give better justification for use of AADL. Compare the different
methodologies with AADL directly. Industrial acceptance can be
considered an important factor.

\subsection{Justification for formal semantics}
The formal semantics chapter is without context. Present its context
in the introductory chapter. State that it \emph{may} be used as the
basis of a systemic verification engine, even though the semantics
have not been implemented yet.

\section{Nomenclature changes}

\begin{itemize}
\item{Change ``inter-dispatch'' to ``inter-arrival'' in the entire
  document, emphasizing that the \emph{arrival} is that of tasks, not
  that of dispatching events, which can arrive at any frequency;}
\item{Correct the MC/DC explanation to align it to that found in
  literature;}
\item{Correct the mixed use of ``model'' and ``design'' in the
  introduction. The ``model'' is one consequence of the \emph{design
    activity};}
\item{In chapter 2, correct the usage of ``dispatch'', align it to the
  definition given in the ARM95 and ARM2005.}
\end{itemize}

\section{Introduction and Bibliography chapters}
In the introductory part, explain in detail what is meant by ``some
real-time systems can be implemented simply by a high resolution
timer''. Improve the section on DO-178B, it is not an SDLC, but rather
a set of constraints leveraged on each part of the SDLC. Improve the
description of the PIM $\to$ PSM $\to$ source code.

First thing to do is extend the bibliography to make it a more
thorough and plausible account of exploratory activity; rather than a
\emph{post-facto} overview of what came before. Also, make a stronger
case for the selection of AADL as the design vehicle. Some minor
points to change include:

\begin{itemize}
\item{Correct parse error in Lustre example;}
\item{SCADE is not a methodology, it is a tool;}
\item{Remove extraneous paragraphing in scheduling subsection;}
\item{State that PCP has nothing to do with preventing livelock;}
\item{Correct statement to say that active priority is the max of base
  priority and ceiling priority, it is actually a max of the active
  priority and the ceiling priority. A corollary is to explain that
  Ravenscar does not preclude a PO procedure calling another PO
  procedure;}
\item{Correct meta-model notation to make it M0-M3 rather than M1-M4;}
\item{Explain the concept of ``vertical distance'' the first time it
  occurs;}
\item{In HRT-URM, explain in detail that the port construct allows
  embedding of protocol information which is its main advantage;}
\item{Extend the section on MARTE, as it is a major industrial
  concern.}
\end{itemize}

\section{Ada Ravenscar and AADL chapter}
\begin{itemize}
\item{Use Ada instead of Ada 95 and Ada 2005 when it is general
  language issue;}
\item{Stop insulting the rendezvous, as it is a basis of theoretical
  study of concurrency;}
\item{Discuss implications of Ravenscar restrictions, rather than just
  restating what has been published in the guide;}
\item{Correct reference to round robin scheduling, it is FIFO within
  priorities, which is not round robin;}
\item{Discuss use of sporadic tasks to implement precedence relations
  between jobs;}
\item{Specify that my approach is meta-model based, so that extra
  restrictions on constructs can be leveraged.}
\end{itemize}

\section{Model to code generation transforms chapter}
Before going on to actual code generation, explain two extra things:

\begin{enumerate}
\item{Place of the approach of code generation in the larger SDLC;}
\item{Analyses that can be leveraged on the model before actual code
  generation can occur;}
\item{At every code generation pattern's presentation, explicitly
  state what the consequences to the runtime's behavior might be
  because of that particular pattern's selection.}
\end{enumerate}

Other changes that are to be made:

\begin{itemize}
\item{Code generation does not reduce errors per se, will make errors
  more systematic;}
\item{Update history of ORK, it has been given to Ada-Core for
  commercialization;}
\item{Section on ``Response-critical tasks'' needs a more critical
  look, what exactly is it good for?}
\end{itemize}

\section{Deterministic communication chapter}
\begin{itemize}
\item{Explain in detail what is meant by ``a sporadic thread may
  change order of execution''. Specifically state that this chapter
  presents the implementation of a data-flow paradigm on top of an
  asynchronous system;}
\item{State that using the DBX connectors imply inclusion of system
  hypotheses in the model. It also implies that non-functional
  properties are assured by the connectors, and it separates the
  concern from functional properties cleanly. Also that the meta-model
  is where assumptions on the system are to be expressed;}
\item{Clarify that MC/DC testing does not apply to temporal aspects.}
\end{itemize}

\section{Formal semantics chapter}
\begin{itemize}
\item{Clarify that Ravenscar is not a set of patterns, rather a set of
  constraints, to meet which an infinite number of patterns may be
  envisaged;}
\item{Explain why the meta-model diagram in this chapter is different
  from the one in Chapter 4;}
\item{In the section on conformant programs, clarify that the ``D'' is
  not the same in the different productions;}
\item{Don't say that ``protected objects do not make calls to other
  protected objects'';}
\item{Clarify how interrupts send events to tasks.}
\end{itemize}

\end{document}
