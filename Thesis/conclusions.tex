\chapter{Conclusions and future work}{``Even if you're on the right
  track,\\you'll get run over if you just sit there.''}{Will Rogers}
\label{chap:conclusions}
Model driven engineering and automatic code generation is a powerful
mechanism for reducing the workload for software development. It is
even more attractive as a methodology when the target domain is
safety-critical and real-time systems. In a nutshell, MDE proposes
that the architecture and design of a system be given in a high-level
language that is then automatically transformed to source code in a
suitable programming language. MDE is particularly attractive for
safety-critical systems for a number of reasons; automatic model to
code transformation ensures the preservation of properties, it reduces
effort which has an impact on the cost, complexity and time needed to
develop a system, and having a high-level model gives the ability to
carry out analyses on the system before it is built.

There are a large number of existing MDE approaches, a few of
which---those specifically targeted to real-time systems---were
presented in Chapter~\ref{chap:biblio}. These include a large class of
formalisms based on UML or an extension thereof, as well as some
proprietary ones such as Lustre/SCADE Suite and Matlab \simu. Each of
these high-level languages provides a certain level of abstraction. 

It was observed that for some, like Simulink and Lustre, the
abstraction level is very high. The design or architecture is very
close to the ``problem space''. In these languages---which seek to
address the problem of developing computer-controlled control
systems---the constructs of the language mirror the methods in which
control algorithms themselves are expressed. The design artifacts are
blocks that perform a transformation on their inputs and provide the
results as outputs. These blocks are connected together via links that
denote the data flows, and which become the inputs upon which the
transformations are carried out and the outputs that result from
them. This is very close to the control loops that are designed using
mathematical functional notations of $y_n = f(x_{n-1}, x_{n-2},
\dots)$. This high level of abstraction, while beneficial for the
specific problem domain within real-time systems, is quite far removed
from the ``solution space''. The solution space in this case being the
runtime entities that actually execute on a physical computer to
approximate the behavior of the system as described in the
model. These runtime entities are the processes and threads, the
message buffers and mutexes, the procedures and their parameters etc.;
in essence, the computing portion of the big picture. Because of this
vertical distance between the architecture and the final source code,
it can be difficult for the designer to realize the implications,
impacts and consequences of design decisions. Also, these approaches
remain quite restrictive as far as the implementation of
general-purpose real-time systems is concerned. It would be a very
tortuous exercise indeed, to build a C$^4$I (Command, Control,
Computers, Communication and Intelligence) system using Simulink or
SCADE Suite.

On the other end of the spectrum are UML based approaches. UML in its
basic vision is for designing software, hence it is tightly coupled
with the object oriented method of software engineering. Its main
entities are classes and relations among those classes. With the newer
versions of UML, there has been a push towards including execution
level artifacts such as threads into the language. This is done by
overloading the basic semantics of the class entity of the
language. This semantic overload leads to considerable confusion and
to counter-intuitive code generation: is an ``active class'' a class
or a thread? Why isn't the code generated for an active class actually
a class? The profiles of UML specifically targeted to the real-time
domain such as HRT-UML and MARTE engage in further semantic overload
by stipulating that stereotypes can specify that a class is not just a
thread, but a periodic thread, or even a shared message buffer, or a
complete executable component.

It was argued in Chapter~\ref{chap:code_gen} that the Architecture
Analysis \& Design Language provides a better level of abstraction. It
is a domain specific language so there is no question of overload. It
allows block structured modeling but uses execution level constructs
at model level such as processes, threads and subprograms. This,
coupled with the ability to use data ports and event (data) ports
provides an attractive level and style of abstraction for real-time
systems.

The Ada programming language, with its strong typing, modularization,
robust runtime with safety features and the inclusion of tasking and
communication primitives at the level of the language makes it an
attractive choice for real-time systems. This, together with the
definition of the Ravenscar Profile for High-integrity Systems---as
outlined in Chapter~\ref{chap:aadlrs}---and its implementation in the
form of the free ORK kernel and the commercial RAVEN kernel from
Aonix, makes for a compelling case for its use in modern systems. In
fact, Pratt \& Whitney have used the RAVEN kernel for the FADEC
(Full-Authority Digital Engine Control) of their new PW6000 jet
engine~\cite{raven-fadec}.

The major thrust of this research work has been the elaboration of
code generation rules from AADL toward Ravenscar-compliant Ada source
code. The main part of this work, as presented in
Chapter~\ref{chap:code_gen}, obviates a lot of the tedious and
error-prone work of coding a framework in Ada that provides the
non-functional substrate for solving the functional part of the
real-time problem at hand. The automatic code generation saves time
and difficult-to-identify errors in the construction of the
framework. The AADL itself is eminently suited to this kind of
development by providing---as stated---direct representations for
processes, threads and subprograms; and also some indirect
abstractions such as ports and connections between threads. This
brings the best of both worlds together and allows architecture
modeling where the designer can see the consequences of his decisions
on the execution entities generated as well as providing abstractions
for conceptual design of applications.

A major issue in writing control systems that have different
functionalities running at different periods is that all communication
between various jobs must be deterministic. While this is easy to
achieve in a synchronous system running on a cyclic executive---case
in point being Lustre/SCADE Suite and Simulink---it can be tricky in
case a full process-based executive is
used. Chapter~\ref{chap:adv_code} presented a mechanism to produce
deterministic communication constructs automatically from an AADL
description. This, coupled with the automatic verification code
generated in LOTOS, is a step in the direction for creating AADL and
Ada Ravenscar based certifiable control systems.

A formal semantics for the generated Ada code running on a
Ravenscar-compliant executive was given in
Chapter~\ref{chap:formal_sem}. The static semantics were given in set
theoretic notation and the dynamic semantics in the form of Structural
Operational Semantics rules, which manipulate and modify the system
data structures depending on the transitions taken to model the
system's evolution over time.

\section{Self-criticism}
In any scientific endeavor, it is easy---and dangerous---to overlook
or ignore the potential disadvantages of what is being
proposed. Therefore, in this section a few criticisms are leveled at
the work that has been presented. The most important facility yet
lacking in the presented approach is that of distribution. There is no
restriction in the Ravenscar computation model that would preclude the
possibility of having different tasks executing on different physical
processors connected via a communication medium. There is also the
possibility of having a partitioned Ravenscar system in the manner of
the ARINC 653 specification~\cite{arinc}. In fact, some work towards
this end has already been done~\cite{tokar@adalett03} using such
partitioned systems with Ada application-level programs.

Distributed Ravenscar application development is already being
explored via the development of a lightweight distribution middleware
called PolyORB~\cite{zalila@ae07}. This middleware, which is
completely Ravenscar-compliant, can be used to write distributed, hard
real-time systems that conform to the Ravenscar Profile's
computational model. Distributed communication is only done via
message passing, never with synchronous RPC in order to preserve the
semantics of the profile.

Another area that is lacking is that of the capability to specify
``real-time transactions''~\cite{cornwell@ae96}. A real-time
transaction is a conceptual notion of the response of a system to an
event, whether that response is completely provided by one thread
within the system, or by a number of cooperating and communicating
threads. In case the response to an event is to be computed by a
number of threads, the transaction is considered to be the entire
control and data flow between the reception of the event, and action
taken by the system till the event is treated and an output at the
final thread is available. Modeling of transactions is important
because it allows the calculation of a response time that is more
systemic than that of a single thread, since now the entire path of
the treatment of the event through the system can be analyzed,
including the attendant computation times, up until the end of
treatment. This notion is especially useful in the case of control
systems, where the ``event'' would be the dispatching of the periodic
thread that starts the control loop's function by reading sensors, and
the final thread is the one(s) that write(s) commands to the
actuator(s).

There are certain restrictions that are inherent to the Ravenscar
Profile itself, such as the absence of RPC communication between
threads, the inability to signal multiple threads simultaneously (this
is a consequence of having at most one entry per protected object and
at most one task waiting on that entry), and no dynamic creation or
destruction of tasks. These restrictions, however, are more of an
advantage than a disadvantage as they ensure safe and robust operation
of the system.

The verification mechanism implemented, as described in
Chapter~\ref{chap:adv_code} could be improved. At the moment, the
scheduler is not explicitly modeled in the LOTOS code generated,
instead, its functionality is subsumed by the LOTOS processes that
model the threads. A cleaner and more easy to understand approach
would be to generate a separate LOTOS process for the scheduler that
sends events to the various threads at the correct points to enable
them for execution.

The formal semantics as described in Chapter~\ref{chap:formal_sem}
have not been implemented in a suitable language such as AsmL, or
indeed any other convenient formalism. For the provided semantics to
be of engineering use---rather than a documentation aid or a purely
academic exercise---they need to be implemented. The uses and
advantages to having an implemented semantics are myriad, and have
already been discussed.

A minor point is that the transformation rules from AADL to RMM within
the ARC code generator have been written in Java. Although this is a
valid method of implementation, currently there are specialized model
transformation languages available that greatly ease the specification
and implementation of such transformations. One such language, which
is implemented for the EMF framework provided with Eclipse, is the
Atlas Transformation Language (ATL)~\cite{jouault@oopsla06}. It was
not used since at the time ARC was started, ATL was still in its alpha
phase and was not very usable.

\section{Future work}
There are a few axes of research and engineering activity that are a
logical sequel to the work under discussion. Among these activities,
some are purely engineering endeavours, others are purely research,
and some activities have an overlap. In this section a few of these
open questions are addressed and possible future avenues of work are
enumerated.

It is patently obvious that the issues mentioned in the previous
section as criticisms of the work presented are open to
improvement. Of these, code generation for distributed Ravenscar
systems and an ARINC 653 compliant real-time operating system together
with code generation tooling is already being pursued within the
author's own research group.

A major axis of exploration will be opened up via the implementation
of the formal semantics of generated Ravenscar systems as
described. This semantics, once implemented in a formal description
language, could potentially be used for verification via model
checking as well as simulation to aid in the analysis of systems while
still at the design stage.

A separate plugin is also envisaged that can provide code reporting and
documentation information for the generated code. The code generation
rules as presented allow for the direct identification of the source
artifact(s) to which a piece of code corresponds, and vice-versa. A
hypertext report generator that can provide a concise traceability
graph between the architecture model artifacts and their resulting
code can be very useful for certification purposes, specially in the
context of the aerospace industry since the DO-178B standard does
stipulate strict traceability requirements.

A major research avenue is the study of whether and how a synchronous
paradigm such as Lustre could be used in conjunction with AADL. It is
possible to envisage that the response code for each thread be written
not in Ada or C, but in the form of a Lustre node. Each time the
thread is dispatched, it would in fact run the main Lustre node that
corresponds to its response code. The input flows to the main node
would be the \texttt{in data port} features and the output flows would
be the \texttt{out data port} features as defined by the thread. This
could lead to a fusion of the architecture description language
approach with the synchronous approach.

The---approximately---inverse of the previous proposal can also be
considered. Compiling a functional model such as that provided by
Simulink towards the Ada programming language to run on top of a
Ravenscar-compliant RTOS can be useful. Collapsing Simulink blocks at
the same ``rate'' into one Ada task of the same period with Simulink
data connections being represented by AADL data ports is feasible. The
thread callbacks in this case would be automatically generated via an
analysis of the blocks that have been collapsed into it.

%%% Local Variables:
%%% mode: latex
%%% mode: flyspell
%%% TeX-master: t
%%% End:
